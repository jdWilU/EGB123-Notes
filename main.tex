\documentclass{report}
\input{preamble}
\input{macros}
\input{letterfonts}
\begin{document}
	
	\thispagestyle{empty}
	\mytitleb{EGB123 Notes}{Jaden Ussher}{2025}
	\newpage% or \cleardoublepage
	\tableofcontents
	\chapter{Week 1}
	
	
	\section{Unit Introduction}
	\subsection{Unit Motivation}
	\textit{Engineers have the task of creating, maintaining and extending the physical infrastructure that allows society to function. This infrastructure has developed over many generations, in response to the needs and demands of society, and has become more complex over time. \textbf{Planning} and \textbf{design} are the key activities that are used, together with management, to bring projects through to successful completion.}
	
	\dfn{Unit Overview}{This unit introduces the knowledge and skills used to undertake site investigations and project assessments as part of infrastructure planning and management activities to meet stakeholders' needs.}

	\subsection{Assessment Overview}
	The units assessments will consist of the following:

	\begin{itemize}
		\item Assessment 1: \textbf{Study Area Investigation} - You will work in a group of four:
		\begin{itemize}
			\item to investigate functions of a real world engineering system at an urban study area. 
			\item with co/leadership of sub tasks preassigned to specific members with contribution by all other members.
			\item 15\% individual across two sub tasks, 15\% group (30\% total)
			\item \textbf{Submission:} Group submits, each individual submits a text response on own contribution (Same Due Date).
		\end{itemize}
		\item Assessment 2: \textbf{Major Project Assessment} - Work in a group of four:
		\begin{itemize}
			\item Prepare a major project assessment that reviews and analyses prescribed technical aspects of two real world major civil engineering projects.
			\item 20\% individual across two sub tasks, 15\% group (35\% total)
			\item \textbf{Submission:} Group submits, each individual submits a text response on own contribution (Same Due Date).
		\end{itemize}
		\item Assessment 3: \textbf{Final Exam} - Individual (35\%).
	\end{itemize}

	\section{Urban Study Area Investigation}
	You will be provided a study area investigation brief that contains backgrounds as well as directions on how to conduct investigation. Will work in a group of 4 to complete:

	\begin{itemize}
		\item 4 items in Sub Task A (Each led by one group member)
		\item 2 items in Sub Task B (Co led by two group members)
		\item 1 Technical report in Sub Task C (All group members contribute).
	\end{itemize}

	\subsection{Investigation}
	The project will involve three sub tasks:

	\begin{enumerate}
		\item \textbf{Topography and Water Utilities}
		\begin{itemize}
			\item Topography
			\item Stormwater
			\item Sewer
			\item Water Reticulation
		\end{itemize}
		\item \textbf{Bridge Approaches and Bridge Structure}
		\begin{itemize}
			\item Bridge Approaches and Abutments
			\item Bridge Structure
		\end{itemize}
		\item \textbf{Engineering Technical Report}
	\end{enumerate}

	For each item in sub task A, the item will ask you to 
	\begin{itemize}
		\item explain the typology of the system with definitions and roles of components. 
		\item Locate certain features on spatial layer documents.
		\item Annotate responses to queries on spatial layer documents.
		\item Interpret the system under investigation and discuss how that system relates to street layout, build form, ease of movement.
	\end{itemize}

	For sub task B, you will be asked to:
	\begin{itemize}
		\item \textbf{B1} Consider bridge approaches and abutments, and how they relate to the street layout and build form. This will be done using six tables, each containing a series of stated forms or stated considerations about bridge approaches and abutments.
		\item \textbf{B2} Consider the bridge structure for a real bridge in the study area. 
	\end{itemize}

	\subsection{Information Sources}
	Throughout the unit, the following sources of information will be used:
	\begin{itemize}
		\item Spatial information on Community maps, nearmap, google street view. 
		\item Government and agency documentation from Austroads. 
	\end{itemize}


	
	\newpage

	\chapter{Week 2}

	\section{Public Utilities}

	\begin{itemize}
		\item Public utilities is where most urban civil engineering activities are focused.
		\item Civil engineers have a \textbf{duty of care} to ensure that their activities do not harm public utilities and do not allow public utilities to pose health, environment and safety risks to the public.
	\end{itemize}
	\textbf{We need to know where public utilities are located so that we:}

	\begin{itemize}
		\item Can engineer in a way that does not damage them.
		\item Can accommodate any necessary changes
		\item dont damage them during construction.
		\item can coordinate with PU system owners and operators.
	\end{itemize}

	\subsection{What are Public Utilities in the Urban Context?}
	Systems of infrastructure that:

	\begin{itemize}
		\item Convey something (e.g. water reticulation) and/or
		\item Change state of something (e.g. voltage transformer)
	\end{itemize}
	\textbf{For community benefit}. Most often, public utilities need to be located in or near public spaces:

	\begin{itemize}
		\item For ease of access for maintenance, operations etc.
		\item To avoid interference with private land uses
	\end{itemize}

	\subsection{Where are Conveyance Public Utilities Located in the Urban Context?}

	Where possible, within a strip of public land such as a:
	\begin{itemize}
		\item Casement (road land) OR
		\item Easement (special purpose land)
	\end{itemize}

	Benefits of locating within public land:
	\begin{itemize}
		\item Makes access easier
		\item Longitudinal configuration of the land often suits conveyance
	\end{itemize}

	But technical aspects sometimes require their installation through private land allotments:
	\begin{itemize}
		\item e.g. some stormwater, sewer due to grade alignment requirements for hydraulics (fluid flow) using gravity
	\end{itemize}

	\subsection{Where are Change of State Public Utilities Located in the Urban Context?}

	May be located within:
	\begin{itemize}
		\item Road casement or an easement
		\item Public land (e.g. park)
		\item Private land allotments
	\end{itemize}
	\textbf{Sometimes may need to be installed within a structure/building.}

	\subsection{Public Utilities with Individual Property Connections}

	\textbf{Electricity connections:}
	\begin{itemize}
		\item Here are 230V twisted-pair wires for individual house connections
	\end{itemize}

	\textbf{Not every property has a connection to every PU system, e.g.}
	\begin{itemize}
		\item Reticulated gas
		\item Hybrid fibre coaxial (HFC) broadband
	\end{itemize}

	\subsection{Public Utilities and Large Land Uses}

	Large land uses such as hospitals, shopping centres, institutional and highrise apartments:
	\begin{itemize}
		\item May have significant numbers of on-site PU systems
		\item And associated on-site management systems
	\end{itemize}

	\subsection{Who owns Public Utilities Located within an Urban Context?}

	\begin{table}[!h]
	\centering
	\begin{tabular}{|l|l|l|}
	\hline
	\textbf{Government agencies} & \textbf{Statutory authorities and GOCs} & \textbf{Private agencies} \\
	\hline
	Local government & Urban Utilities & TPG \\
	e.g. Brisbane City Council & • sewer, water reticulation, gas & • optic fibre \\
	- stormwater... & & \\
	\hline
	State government & Energex & Transurban \\
	e.g. Transport and Main Roads & • electricity reticulation & • PU on their toll-roads \\
	- some road PU & & \\
	\hline
	\end{tabular}
	\end{table}

	\subsection{Examples of Public Utilities in the Urban Context}

	\begin{table}[h!]
	\centering
	\caption{Public Utilities Examples - Conveyance and Change of State}
	\renewcommand{\arraystretch}{1.5}
	\small
	\begin{tabular}{|>{\bfseries}l|>{\bfseries}p{2.5cm}|p{3.5cm}|p{2cm}|p{3.5cm}|p{2cm}|}
	\toprule
	\rowcolor{blue!30}
	\multicolumn{2}{|c|}{\textbf{Type}} & \textbf{Example Components} & \textbf{Ownership} & \textbf{Location} & \textbf{Position} \\
	\midrule
	\multicolumn{6}{|c|}{\cellcolor{blue!15}\textbf{CONVEYANCE PUBLIC UTILITIES}} \\
	\midrule
	\multirow{5}{*}{\rotatebox[origin=c]{90}{\textbf{Conveyance}}} 
	& Sewer & Pipe Connector, Maintenance Shaft & Urban Utilities & Within road casements and/or easements and/or private properties & Underground \\
	\cline{2-6}
	& Stormwater & Kerb and channel, Pipe & Brisbane City Council & Within road casements and/or public land and/or private properties & At-grade and underground \\
	\cline{2-6}
	& Electricity & Ducted cable, Overhead wires on poles & Energex & Within road casements and/or easements and/or public land & Underground or overhead \\
	\cline{2-6}
	& Telecoms (copper) & Ducted cable, Pit & Telstra & Within road casements and/or easements & Underground \\
	\cline{2-6}
	& Telecoms (fibre) & Ducted cable, Pit & Optus & Within road casements and/or easements & Underground \\
	\midrule
	\multicolumn{6}{|c|}{\cellcolor{green!15}\textbf{CHANGE OF STATE PUBLIC UTILITIES}} \\
	\midrule
	\multirow{4}{*}{\rotatebox[origin=c]{90}{\textbf{Change}}} 
	& Sewer & Sewage pump station & Urban Utilities & Within public land (park at catchment low point) & Generally at-grade \\
	\cline{2-6}
	& Stormwater & Stormwater Quality Improvement Device (SQID) & Brisbane City Council & Within watercourse boundary (state public land) & At-grade \\
	\cline{2-6}
	& Electricity & Substation & Energex & Within a land allotment & At-grade \\
	\cline{2-6}
	& Telecoms & Exchange & Telstra & Within land allotment & At-grade \\
	\bottomrule
	\end{tabular}
	\end{table}
	
	
	\newpage
	\subsection{How do we know where Public Utilities are located?}

	\begin{itemize}
		\item \textbf{Community Maps} - Brisbane City Council provides a community maps service that allows you to view the location of public utilities in Brisbane.
		\item \textbf{Nearmap} - An aerial imagery service that provides high-resolution images of urban areas, which can be used to identify public utilities.
		\item \textbf{Google Street View} - A service that allows you to view street-level imagery, which can help in identifying the location of public utilities.
	\end{itemize}

	
	\section{Geospatial Mapping}

	\dfn{What is Geospatial Mapping?}{A spatial visualisation method that enables the creation of customised maps to address specific requirements. Its primary aim is to show items with geographic coordinates in a geographical framework, providing a representation of the physical world on a map. Various approaches, solutions, and Geographic Information Systems (GIS) software can be employed to analyze existing geospatial data and geographical and terrestrial databases.}

	\subsection{Community Maps}
	Community maps are a geospatial mapping service provided by the Brisbane City Council that provides

	\thm{Theorem Name}{Theorem Statement}
	\cor[cori]{Corollary Name}{Corollary Statement}
	\lem{Lemma Name}{Lemma Statement}
	\clm{Claim Name}{Claim Statement}
	\ex{Example Name}{Example explained}


	\opn{Open Question Name}{Question Statement}
%\qs{Question Name}{Question Statement}
\nt{Special Note}
\wc{Wrong Concept topic}{Explanation}
%\pf{Proof}{Proof}
	
\chapter{Second Chapter}
\section{Section 1}
\end{document}